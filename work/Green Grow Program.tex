\documentclass[12pt]{article}

\usepackage[margin=1in]{geometry}
\usepackage{amsmath}
\usepackage{booktabs}
\usepackage{array}
\usepackage{graphicx}

\begin{document}
\section*{Summary of the Key Facts}

\textbf{Company:} Try Recycling \& Aggregates (Try) \\
\textbf{Location:} London, Ontario \\
\textbf{Founded:} 1991 by Bill Graham \\
\textbf{President:} Jim Graham (since 1999) \\
\textbf{Core Business:} Recycling and repurposing construction/landscape materials into new products \\

\noindent
\textbf{Revenue Sources:}
\begin{itemize}
    \item Tipping fees (65\% local, 35\% American customers)
    \item Sale of recycled products
\end{itemize}

\noindent
\textbf{Products \& Initiatives}
\begin{itemize}
    \item \textbf{Products:} Compost, landscape mulch, two-way topsoil, tri-mix topsoil, landscape woodchips, etc.
    \item \textbf{Community Involvement:} Nature park initiatives, ``Ethics in Action'' award, involvement in London Composts
\end{itemize}

\noindent
\textbf{Retail Shift}
\begin{itemize}
    \item Transitioning from serving commercial clients at a ``destination site'' to attracting residential customers 
    \item Self-serve facility launched in 2000; retail sales accounted for 5\% of 2000 gross revenues
\end{itemize}

\section*{The Grow Green Program}

\textbf{Objective:} A fundraising partnership program offering Try’s gardening products to schools, churches, sports teams, and nonprofits. \\
\textbf{Pricing:}
\begin{itemize}
    \item End consumers pay regular retail.
    \item Fundraising organizations pay wholesale.
    \item Delivery is included.
\end{itemize}

\noindent
\textbf{Expected Product Mix:} Mirrors the existing self-serve product split. \\
\textbf{Profit Goal:} \$35{,}000.

\section*{Target Customers}

\begin{itemize}
    \item \textbf{Local schools:} Over 80 in London; often in need of fundraising support
    \item \textbf{Community groups:} Churches, sports teams, nonprofits with a single coordinator for fundraising
\end{itemize}

\section*{Competition \& Fundraising Landscape}

\begin{itemize}
    \item Other programs offer chocolates, magazines, gourmet food, coffee, cheese, etc.
    \item Typical profit margins for groups: 30\% to 55\%.
    \item Common incentives: free delivery, prizes, sorting, promotional materials.
\end{itemize}

\section*{Key Considerations for the Decision}

\subsection*{1. Market Timing \& Demand}
\begin{itemize}
    \item \textbf{Spring Gardening Season:} Timing is crucial; gardening supplies peak in the spring.
    \item \textbf{Growing Environmental Awareness:} London residents are highly engaged in waste reduction \& recycling.
\end{itemize}

\subsection*{2. Financial Feasibility}
\begin{itemize}
    \item \textbf{Target Profit (\$35,000):} Must evaluate the sales volume required at wholesale pricing.
    \item \textbf{Marketing Costs:} About \$500 per organization for print materials, samples, brochures. Delivery is included, increasing cost commitments.
\end{itemize}

\subsection*{3. Fit with Company Capabilities}
\begin{itemize}
    \item \textbf{Production \& Logistics:} Can current operations handle large or sudden demand spikes for packaged soils, mulch, compost, etc.?
    \item \textbf{Deliveries:} Are trucks, drivers, and dispatch systems sufficient for multiple small drop-offs?
    \item \textbf{Expertise:} Try’s background in recycling aligns well with a product-based fundraising program.
\end{itemize}

\subsection*{4. Competition \& Differentiation}
\begin{itemize}
    \item \textbf{Fundraising ``Clutter'':} Schools and nonprofits are inundated with product-based fundraisers.
    \item \textbf{Value Proposition:} Emphasizing local, eco-friendly, practical products can stand out.
    \item \textbf{Margins for Groups:} Must be compelling (30--55\%) to attract participants.
\end{itemize}

\subsection*{5. Brand \& Community Impact}
\begin{itemize}
    \item \textbf{Reputation:} Building on Try’s community involvement enhances goodwill.
    \item \textbf{Educational Angle:} Showcasing recycling and ``green'' initiatives resonates with schools and nonprofits.
\end{itemize}

\subsection*{6. Risks \& Mitigation}
\begin{itemize}
    \item \textbf{Sales Uncertainty:} A new program with no prior track record.
    \item \textbf{Inventory \& Overhead:} Upfront costs may not be fully recovered if participation lags.
    \item \textbf{Implementation Complexity:} Requires coordination with multiple organizations.
\end{itemize}

\section*{Analysis (From the Assistant's Prior Answer)}

\subsection*{1. Favorable Market Trends}
\begin{itemize}
    \item \textbf{High Recycling Participation:} About 90\% of London residents recycle; a growing network of eco-conscious groups.
    \item \textbf{Spring Demand:} Seasonal alignment with gardening/home landscaping projects.
    \item \textbf{Fundraising Needs:} Schools and nonprofits are open to creative, responsible fundraising methods.
\end{itemize}

\subsection*{2. Corporate Alignment \& Reputation}
\begin{itemize}
    \item \textbf{Established CSR Track Record:} Previous community projects and awards add credibility.
    \item \textbf{Brand Synergy:} A program offering green products aligns with Try’s mission and identity.
\end{itemize}

\subsection*{3. Profit Potential}
\begin{itemize}
    \item \textbf{Target Profit = \$35,000:} With proper volume, the margin spread between wholesale prices and variable costs should enable this goal.
    \item \textbf{Support Costs (\$500/org):} Recoupable through sufficient sales; partnering with multiple groups can offset overhead.
\end{itemize}

\subsection*{4. Low Risk of Cannibalization}
\begin{itemize}
    \item \textbf{Minimal Retail Erosion:} Retail sales comprised only 5\% of 2000 revenue, so wholesale fundraising may bring in new buyers.
\end{itemize}

\subsection*{5. Differentiation in Fundraising Space}
\begin{itemize}
    \item \textbf{Eco-Friendly Offering:} Competes favorably with typical candy/magazine fundraisers.
\end{itemize}

\section*{Potential Challenges (From the Assistant's Prior Answer)}

\begin{itemize}
    \item \textbf{Logistics \& Delivery Costs:} Offering ``delivery included'' can strain margins if routes are inefficient.
    \item \textbf{Fundraising Management Complexity:} Approval needed from principals or board committees could delay uptake.
    \item \textbf{Upfront Marketing Expenditure:} \$500 per organization could be costly if sign-ups or sales volumes are low.
    \item \textbf{Sales Volume Uncertainty:} Falling short of target volumes could undercut profits.
\end{itemize}

\section*{Conclusion: Is the Time Right?}

\textbf{Favorable Indicators:}
\begin{itemize}
    \item \textbf{Market Climate:} High environmental awareness and a strong culture of recycling in London.
    \item \textbf{Seasonal Alignment:} Spring is the prime season for selling gardening products.
    \item \textbf{Community Engagement:} Try’s community-focused initiatives and positive brand reputation.
\end{itemize}

\noindent
\textbf{Potential Obstacles:}
\begin{itemize}
    \item \textbf{Upfront Costs:} Marketing materials, delivery, and production overhead could strain resources if demand falls short.
    \item \textbf{Competition from Familiar Fundraising Options:} Many schools and nonprofits already rely on established, well-known products.
\end{itemize}

\noindent
\textbf{Overall Assessment:} 
The Grow Green Program aligns well with Try Recycling’s mission and capabilities while leveraging strong market momentum around eco-friendly products. Although there are risks, notably around financial uncertainty and competition, the spring gardening season presents a timely opportunity. Jim Graham’s \$35,000 profit goal seems achievable if the program is managed carefully, especially regarding delivery logistics and ensuring a compelling fundraising margin for participant groups.

\section*{Final Recommendation}

\textbf{Yes: Launch the Grow Green Program}, but implement a carefully managed pilot phase:
\begin{itemize}
    \item Partner with a limited number of schools or community groups initially to gauge demand and refine logistics.
    \item Monitor delivery schedules and marketing costs to ensure they do not erode margins.
    \item Adjust pricing or minimum order requirements if necessary to secure sufficient profitability for both Try and the participant organizations.
\end{itemize}

\noindent
If the pilot proves successful, scale up and refine the program for subsequent seasons. In doing so, Try Recycling can capitalize on its strong community reputation, fulfill its \$35{,}000 profit target, and offer a uniquely green alternative in a crowded fundraising market.
\section*{Comprehensive Solutions for Try Recycling \& Aggregates Case}

\textbf{Questions}
\begin{enumerate}
    \item Contribution margin rate for each product (Try \& participating organizations).
    \item Weighted average contribution margin rate (Try \& participating organizations).
    \item Breakeven sales (no-profit) per organization and target-profit (\$35,000) breakeven for 1, 10, and 25 organizations.
    \item Number of customers needed to meet target-profit breakeven sales (1, 10, and 25 organizations).
\end{enumerate}

\section{1) Contribution Margin Rate for Each Product}

\subsection*{Data from the Case}

\begin{table}[h!]
\centering
\begin{tabular}{lccc}
\toprule
\textbf{Product} & \textbf{Retail Price (R)} & \textbf{Wholesale Price (W)} & \textbf{Variable Cost (VC)} \\
\midrule
Compost             & \$18.00  & \$13.00 & \$11.00 \\
Landscape Mulch     & \$12.50  & \$10.00 & \$4.00  \\
Two-Way Topsoil     & \$19.00  & \$12.00 & \$8.50  \\
Tri-Mix Topsoil     & \$22.00  & \$17.00 & \$11.75 \\
Landscape Woodchips & \$32.00  & \$27.00 & \$18.00 \\
\bottomrule
\end{tabular}
\end{table}

\subsection*{A) Try’s Contribution Margin Rate (Wholesale Prices)}

\[
\text{CM per yard (Try)} = W - VC, 
\quad
\text{CM Rate} = \frac{(W - VC)}{W} \times 100\%
\]

\begin{table}[h!]
\centering
\begin{tabular}{lcccc}
\toprule
\textbf{Product} & \boldmath{$W$} & \boldmath{$VC$} & \textbf{CM} (\$\,\textit{W} - \textit{VC}) & \textbf{CM Rate} \\
\midrule
Compost             & \$13.00 & \$11.00 & \$2.00  & $ \frac{2.00}{13.00} = 15.38\% $ \\
Landscape Mulch     & \$10.00 & \$4.00  & \$6.00  & $ \frac{6.00}{10.00} = 60.00\% $ \\
Two-Way Topsoil     & \$12.00 & \$8.50  & \$3.50  & $ \frac{3.50}{12.00} = 29.17\% $ \\
Tri-Mix Topsoil     & \$17.00 & \$11.75 & \$5.25  & $ \frac{5.25}{17.00} = 30.88\% $ \\
Landscape Woodchips & \$27.00 & \$18.00 & \$9.00  & $ \frac{9.00}{27.00} = 33.33\% $ \\
\bottomrule
\end{tabular}
\end{table}

\subsection*{B) Fundraising Organizations’ Contribution Margin Rate (Retail Prices)}

\[
\text{CM per yard (Orgs)} = R - W, 
\quad
\text{CM Rate} = \frac{(R - W)}{R} \times 100\%
\]

\begin{table}[h!]
\centering
\begin{tabular}{lcccc}
\toprule
\textbf{Product} & \boldmath{$R$} & \boldmath{$W$} & \textbf{CM} (\$\,\textit{R} - \textit{W}) & \textbf{CM Rate} \\
\midrule
Compost             & \$18.00 & \$13.00 & \$5.00  & $ \frac{5.00}{18.00} = 27.78\% $ \\
Landscape Mulch     & \$12.50 & \$10.00 & \$2.50  & $ \frac{2.50}{12.50} = 20.00\% $ \\
Two-Way Topsoil     & \$19.00 & \$12.00 & \$7.00  & $ \frac{7.00}{19.00} = 36.84\% $ \\
Tri-Mix Topsoil     & \$22.00 & \$17.00 & \$5.00  & $ \frac{5.00}{22.00} = 22.73\% $ \\
Landscape Woodchips & \$32.00 & \$27.00 & \$5.00  & $ \frac{5.00}{32.00} = 15.63\% $ \\
\bottomrule
\end{tabular}
\end{table}

\clearpage

\section{2) Weighted Average CM Rate (Try \& Organizations)}

\subsection*{Expected Product Mix}

\begin{table}[h!]
\centering
\begin{tabular}{lc}
\toprule
\textbf{Product} & \textbf{Expected \% of Units Sold} \\
\midrule
Compost             & 12\% \\
Landscape Mulch     & 8\%  \\
Two-Way Topsoil     & 51\% \\
Tri-Mix Topsoil     & 9\%  \\
Landscape Woodchips & 20\% \\
\bottomrule
\end{tabular}
\end{table}

\subsection*{A) Try’s Weighted Average CM Rate}

\begin{align*}
\text{Compost} &:\quad 15.38\% \times 0.12 = 0.018456,\\
\text{Landscape Mulch} &:\quad 60.00\% \times 0.08 = 0.0480,\\
\text{Two-Way Topsoil} &:\quad 29.17\% \times 0.51 = 0.148767,\\
\text{Tri-Mix Topsoil} &:\quad 30.88\% \times 0.09 = 0.027792,\\
\text{Woodchips} &:\quad 33.33\% \times 0.20 = 0.06666.\\
\\
\text{Total} & = 0.018456 + 0.0480 + 0.148767 + 0.027792 + 0.06666 \approx 0.309675
\end{align*}

\[
\boxed{30.97\% \text{ (Weighted Avg.\ CM Rate for Try)}}
\]

\subsection*{B) Organizations’ Weighted Average CM Rate}

\begin{align*}
\text{Compost} &:\quad 27.78\% \times 0.12 = 0.033336,\\
\text{Landscape Mulch} &:\quad 20.00\% \times 0.08 = 0.016,\\
\text{Two-Way Topsoil} &:\quad 36.84\% \times 0.51 = 0.187884,\\
\text{Tri-Mix Topsoil} &:\quad 22.73\% \times 0.09 = 0.020457,\\
\text{Woodchips} &:\quad 15.63\% \times 0.20 = 0.03126.\\
\\
\text{Total} & = 0.033336 + 0.016 + 0.187884 + 0.020457 + 0.03126 \approx 0.288937
\end{align*}

\[
\boxed{28.89\% \text{ (Weighted Avg.\ CM Rate for Organizations)}}
\]

\section{3) Breakeven Sales per Org \& Target-Profit (\$35,000) Breakeven}

We use \textbf{Try’s weighted average CM rate (30.97\%)} for these calculations.

\subsection*{Assumptions}
\begin{enumerate}
    \item Fixed Cost per Org = \$500 (marketing materials, brochures, etc.).
    \item Overall Profit Goal for the entire Grow Green Program = \$35,000.
\end{enumerate}

\subsection*{(a) Breakeven (No Profit) Sales per Organization}

At breakeven, the contribution margin from each organization just covers that organization’s fixed cost.

\[
\text{Breakeven Sales per Org} 
= \frac{\text{Fixed Cost}}{\text{CM Rate}} 
= \frac{500}{0.3097} 
\approx \$1{,}615
\]

\subsection*{(b) Target-Profit Breakeven (Total = \$35,000)}

We must cover:
\begin{itemize}
    \item Fixed costs = \$500 $\times$ (\# of orgs).
    \item Desired profit = \$35,000.
\end{itemize}
We then divide by the CM rate (0.3097) to find total sales (for all orgs), then determine the per-organization requirement.

\subsubsection*{Scenario 1: 1 Organization}
\begin{itemize}
    \item Total fixed cost = \$500
    \item Desired profit = \$35,000
    \item Total CM needed = 500 + 35{,}000 = \$35{,}500
\[
\text{Total Sales} 
= \frac{35{,}500}{0.3097} 
\approx \$114{,}600
\]
\item \textbf{Sales per organization} = \$114{,}600 (only 1 org).
\end{itemize}

\subsubsection*{Scenario 2: 10 Organizations}
\begin{itemize}
    \item Total fixed cost = 10 $\times$ \$500 = \$5{,}000
    \item Desired profit = \$35{,}000
    \item Total CM needed = 5{,}000 + 35{,}000 = \$40{,}000
\[
\text{Total Sales} 
= \frac{40{,}000}{0.3097} 
\approx \$129{,}200
\]
\item \textbf{Per-organization} = \$129{,}200 / 10 = \$12{,}920
\end{itemize}

\subsubsection*{Scenario 3: 25 Organizations}
\begin{itemize}
    \item Total fixed cost = 25 $\times$ \$500 = \$12{,}500
    \item Desired profit = \$35{,}000
    \item Total CM needed = 12{,}500 + 35{,}000 = \$47{,}500
\[
\text{Total Sales} 
= \frac{47{,}500}{0.3097} 
\approx \$153{,}200
\]
\item \textbf{Per-organization} = \$153{,}200 / 25 = \$6{,}130
\end{itemize}

\subsection*{Summary: Breakeven Sales (No Profit vs.\ Target Profit)}
\begin{itemize}
    \item \textbf{Breakeven (No Profit) Sales/Org} = \(\approx \$1{,}615\)
    \item \textbf{Target Profit (\$35{,}000) Sales}:
    \begin{itemize}
        \item 1 Org: \$114{,}600
        \item 10 Orgs: \$12{,}920 per org (total \$129{,}200)
        \item 25 Orgs: \$6{,}130 per org (total \$153{,}200)
    \end{itemize}
\end{itemize}

\clearpage

\section{4) Number of Customers Needed for Target-Profit Breakeven}

\subsection*{Overview}
\begin{enumerate}
    \item Total Sales Needed at wholesale (from \S 3).
    \item Weighted Average Wholesale Price per yard = \$15.41.
    \item Assume 1 yard per customer (for simplicity).
    \item \# of customers = (Total Sales at Wholesale) / (Weighted Avg.\ Wholesale Price).
\end{enumerate}

\subsection*{Weighted Average Wholesale Price Calculation}

\[
(13 \times 0.12) + (10 \times 0.08) + (12 \times 0.51) + (17 \times 0.09) + (27 \times 0.20) 
= 1.56 + 0.80 + 6.12 + 1.53 + 5.40 
= 15.41
\]

\subsection*{Scenarios}

\subsubsection*{1 Organization}
\begin{itemize}
    \item Total sales needed: \$114{,}600
    \item Wtd.\ avg.\ wholesale price: \$15.41
\[
\text{Customers} 
= \frac{114{,}600}{15.41} 
\approx 7{,}440
\]
\end{itemize}

\subsubsection*{10 Organizations}
\begin{itemize}
    \item Total sales (all 10 orgs): \$129{,}200
\[
\text{Customers (total)} 
= \frac{129{,}200}{15.41} 
\approx 8{,}386
\]
\item Per-organization average: \(8{,}386 \div 10 \approx 839\)
\end{itemize}

\subsubsection*{25 Organizations}
\begin{itemize}
    \item Total sales (all 25 orgs): \$153{,}200
\[
\text{Customers (total)} 
= \frac{153{,}200}{15.41} 
\approx 9{,}942
\]
\item Per-organization average: \(9{,}942 \div 25 \approx 398\)
\end{itemize}

\subsection*{Final Customer Summary}

\begin{table}[h!]
\centering
\begin{tabular}{lcccc}
\toprule
\textbf{Scenario} & \textbf{Total Sales Needed} & \textbf{Avg.\$/Yard} & \textbf{Total Customers} & \textbf{Cust.\ / Org} \\
\midrule
1 Organization           & \$114,600 & \$15.41 & 7,440 & 7,440 \\
10 Organizations (total) & \$129,200 & \$15.41 & 8,386 & 839   \\
25 Organizations (total) & \$153,200 & \$15.41 & 9,942 & 398   \\
\bottomrule
\end{tabular}
\end{table}

\noindent
\textit{\textbf{Note:}} This assumes exactly 1 yard per customer. In reality, some customers may purchase multiple yards, so the actual number of people needed could be lower, depending on typical purchase size.

\end{document}
